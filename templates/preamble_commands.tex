\renewcommand{\bfdefault}{bx}

\hypersetup{
    colorlinks,
    linkcolor={linkcolor},
    citecolor={citecolor}, 
    urlcolor={urlcolor}
}

% Lengths and widths
\addtolength{\textwidth}{6cm}
\addtolength{\textheight}{-1cm}
\addtolength{\hoffset}{-3cm}
\addtolength{\voffset}{-2cm}
\setlength{\tabcolsep}{0.2cm} % Space between columns
\setlength{\headsep}{-12pt} % Reduce space between header and content
\setlength{\headheight}{85pt} % If less, LaTeX automatically increases it
\renewcommand{\footrulewidth}{0pt} % Remove footer line
\renewcommand{\headrulewidth}{0pt} % Remove header line
%\renewcommand{\seqinsert}{\ifmmode\allowbreak\else\-\fi} % Hyphens in seqsplit
% This two commands together give roughly
% the right line height in the tables
\renewcommand{\arraystretch}{1.3}
\onehalfspacing

% Commands
\newcommand{\SetRowColor}[1]{\noalign{\gdef\RowColorName{#1}}\rowcolor{\RowColorName}} % Shortcut for row colour
%\newcommand{\mymulticolumn}[3]{\multicolumn{#1}{>{\columncolor{\RowColorName}}#2}{#3}} % For coloured multi-cols
\newcommand{\colouredMulticolumn}[2]{\multicolumn{#1}{>{\columncolor{\RowColorName}}x{0.9\linewidth}}{#2}} % For coloured multi-cols
\newcolumntype{x}[1]{>{\raggedright}p{#1}} % New column types for ragged-right paragraph columns
\newcommand{\norm}[1]{\left\lVert#1\right\rVert}
\newcommand*{\suchthat}{\;\ifnum\currentgrouptype=16 \middle\fi|\;}
\newcommand{\textasterisk}{*}
\DeclareMathOperator{\Tr}{Tr}

% Centering part/sections
\allsectionsfont{\centering}

\newtcblisting{code}[2]{%
  listing engine=minted,
  colback=lightbackground,
  colframe=darkbackground,
  listing only,
  %breakable,
  minted language=#1,
  minted style=perldoc,
  minted options={linenos=false,texcl=true,breaklines=true, fontsize=\scriptsize, escapeinside=\%\% ,mathescape=true},
  % see https://github.com/gpoore/minted/issues/179 for escape issues
  left=1mm,
  title=\textbf{\textcolor{textcolor}{\faCode}\textsubscript{\footnotesize\textcolor{textcolor}{\texttt{#1}}}~#2}
}

%\NewDocumentCommand{\piechart}{m O{text=inside, sum=auto, explode = 0.1}}{
%  \resizebox{0.9\linewidth}{!}{
%    \begin{tikzpicture}
%      \pie [#2]{#1}
%    \end{tikzpicture}
%    }
%}
\newcommand{\pieChart}[2][1]{
  \vspace*{0.3em}
  \resizebox{\linewidth}{!}{
      \begin{tikzpicture}

      \pgfkeys{/pgf/number format/.cd,
      fixed,fixed zerofill,precision=0,
      set thousands separator={\,}}

        \pie [sum=auto, text=legend, explode = 0.1, scale numbers=#1]{#2}
      \end{tikzpicture}
    }
}

\newcommand{\image}[2][]{%
\begin{center}
\tcbox[colframe=darkbackground, enhanced jigsaw, boxsep=0pt, top=0pt, bottom=0pt, left=0pt, right=0pt, 
boxrule=0.4pt, drop fuzzy shadow, clip upper,
toptitle=2pt,bottomtitle=2pt,nobeforeafter,
title=~~\textbf{\textcolor{textcolor}{\faImage}~#1}]
{\includegraphics[width=\linewidth-4mm]{#2}}
\end{center}}

\newtcolorbox{quotes}{%
    enhanced jigsaw, 
    colback=lightbackground,
    fontupper=\color{textcolor},
    fontlower=\color{textcolor},
    %breakable,      % allow page breaks
    frame hidden,   % hide the default frame
    left=0.5cm,       % left margin
    right=0.5cm,      % right margin
    overlay={%
        \node [scale=8,
            text=black,
            inner sep=0pt,] at ([xshift=0.1cm,yshift=-1cm]frame.north west){``}; 
        \node [scale=8,
            text=black,
            inner sep=0pt,] at ([xshift=-0.1cm]frame.south east){''};  
            },
        % paragraph skips obeyed within tcolorbox
                parbox=false,
}


% 0.0
\newenvironment{oneColumn}
{\tabularx{0.9\linewidth}{X}}
{\endtabularx}


% (5.377 - 0.4) / 2
\newenvironment{twoColumns}
{\tabularx{0.9\linewidth}{x{0.45\linewidth - 0.2cm} x{0.45\linewidth - 0.2cm}}}
{\endtabularx}

% (5.377 - 0.8) / 3
\newenvironment{threeColumns}
{\tabularx{0.9\linewidth}{p{0.3\linewidth - 0.266cm} p{0.3\linewidth- 0.267cm} p{0.3\linewidth- 0.267cm}}}
{\endtabularx}

% (5.377 - 1.2) / 4
\newenvironment{fourColumns}
{\tabularx{0.9\linewidth}{p{0.225\linewidth- 0.3cm} p{0.225\linewidth - 0.3cm} p{0.225\linewidth - 0.3cm} p{0.225\linewidth - 0.3cm}}}
{\endtabularx}

\newenvironment{plainTex}
{\tabularx{0.9\linewidth}{X}}
{\endtabularx}

\newenvironment{algo}
{\tabularx{0.9\linewidth}{X}}
{\endtabularx}

\newenvironment{faq}
{\tabularx{0.9\linewidth}{X}}
{\endtabularx}

%\newenvironment{piechart}
%{\tabularx{0.9\linewidth}{X}}
%{\endtabularx}

\newenvironment{vtime}
{\tabularx{0.9\linewidth}{X}}
{\endtabularx}

\newenvironment{barChart}
{\tabularx{0.9\linewidth}{p{0.3\linewidth - 0.266cm} p{0.3\linewidth- 0.267cm} p{0.3\linewidth- 0.267cm}}}
{\endtabularx}


%%%%%%%%%%%%%% Timeline %%%%%%%%%%%%%

%%% from https://tex.stackexchange.com/a/197447/125774

% code by Andrew:
% https://tex.stackexchange.com/a/28452/13304
\makeatletter
\let\matamp=&
\catcode`\&=13
\makeatletter
\def&{\iftikz@is@matrix
  \pgfmatrixnextcell
  \else
  \matamp
  \fi}
\makeatother

\newcounter{lines}
\def\endlr{\stepcounter{lines}\\}

\newcounter{vtml}
\setcounter{vtml}{0}

\newif\ifvtimelinetitle
\newif\ifvtimebottomline
\tikzset{description/.style={
  column 2/.append style={#1}
 },
 timeline color/.store in=\vtmlcolor,
 timeline color=red!80!black,
 timeline color st/.style={fill=\vtmlcolor,draw=\vtmlcolor},
 use timeline header/.is if=vtimelinetitle,
 use timeline header=false,
 add bottom line/.is if=vtimebottomline,
 add bottom line=false,
 timeline title/.store in=\vtimelinetitle,
 timeline title={},
 line offset/.store in=\lineoffset,
 line offset=4pt,
}

\NewEnviron{vtimeline}[1][]{%
\setcounter{lines}{1}%
\stepcounter{vtml}%
\begin{tikzpicture}[column 1/.style={anchor=east},
 column 2/.style={anchor=west,text width=4.377cm},
 text depth=0pt,text height=1ex,
 row sep=5ex,
 column sep=1em,
 #1
]
\matrix(vtimeline\thevtml)[matrix of nodes]{\BODY};
\pgfmathtruncatemacro\endmtx{\thelines-1}
\path[timeline color st] 
($(vtimeline\thevtml-1-1.north east)!0.5!(vtimeline\thevtml-1-2.north west)$)--
($(vtimeline\thevtml-\endmtx-1.south east)!0.5!(vtimeline\thevtml-\endmtx-2.south west)$);
\foreach \x in {1,...,\endmtx}{
 \node[circle,timeline color st, inner sep=0.15pt, draw=white, thick] 
 (vtimeline\thevtml-c-\x) at 
 ($(vtimeline\thevtml-\x-1.east)!0.5!(vtimeline\thevtml-\x-2.west)$){};
 \draw[timeline color st](vtimeline\thevtml-c-\x.west)--++(-3pt,0);
 }
 \ifvtimelinetitle%
  \draw[timeline color st]([yshift=\lineoffset]vtimeline\thevtml.north west)--
  ([yshift=\lineoffset]vtimeline\thevtml.north east);
  \node[anchor=west,yshift=16pt,font=\large]
   at (vtimeline\thevtml-1-1.north west) 
   {\textsc{Timeline \thevtml}: \textit{\vtimelinetitle}};
 \else%
  \relax%
 \fi%
 \ifvtimebottomline%
   \draw[timeline color st]([yshift=-\lineoffset]vtimeline\thevtml.south west)--
  ([yshift=-\lineoffset]vtimeline\thevtml.south east);
 \else%
   \relax%
 \fi%
\end{tikzpicture}
}

%%%%%%%%%%%%%% Calendar %%%%%%%%%%%%%%%%%%%%%
\makeatletter
% Define our own style
\tikzstyle{week list sunday}=[
  % Note that we cannot extend from week list,
  % the execute before day scope is cumulative
  execute before day scope={%
    \ifdate{day of month=1}{\ifdate{equals=\pgfcalendarbeginiso}{}{
        % On first of month, except when first date in calendar.
        \pgfmathsetlength{\pgf@y}{\tikz@lib@cal@month@yshift}%
        \pgftransformyshift{-\pgf@y}
      }}{}%
  },
  execute at begin day scope={%
    % Because for TikZ Monday is 0 and Sunday is 6,
    % we can't directly use \pgfcalendercurrentweekday,
    % but instead we define \c@pgf@counta (basically) as:
    % (\pgfcalendercurrentweekday + 1) % 7
    \pgfmathsetlength\pgf@x{\tikz@lib@cal@xshift}%
    \ifnum\pgfcalendarcurrentweekday=6
    \c@pgf@counta=0
    \else
    \c@pgf@counta=\pgfcalendarcurrentweekday
    \advance\c@pgf@counta by 1
    \fi
    \pgf@x=\c@pgf@counta\pgf@x
    % Shift to the right position for the day.
    \pgftransformxshift{\pgf@x}
  },
  execute after day scope={
    % Week is done, shift to the next line.
    \ifdate{Saturday}{
      \pgfmathsetlength{\pgf@y}{\tikz@lib@cal@yshift}%
      \pgftransformyshift{-\pgf@y}
    }{}%
  },
  % This should be defined, glancing from the source code.
  tikz@lib@cal@width=7
]
\tikzoption{day headings}{\tikzstyle{day heading}=[#1]}
\tikzstyle{day heading}=[]
\tikzstyle{day letter headings}=[
  execute before day scope={\ifdate{day of month=1}{%
      \pgfmathsetlength{\pgf@ya}{\tikz@lib@cal@yshift}%
      \pgfmathsetlength\pgf@xa{\tikz@lib@cal@xshift}%
      \pgftransformyshift{-\pgf@ya}
      \foreach \d/\l in {0/D,1/S,2/T,3/Q,4/Q,5/S,6/S} {
        \pgf@xa=\d\pgf@xa%
        \pgftransformxshift{\pgf@xa}%
        \pgftransformyshift{\pgf@ya}%
        \node[every day, day heading]{\tiny\l};%
      } 
    }{}%
  }%
]
\makeatother

\tikzstyle{labest}=[font=\footnotesize, fill=orange!50, inner sep=2pt]
\tikzstyle{evento}=[font=\footnotesize, fill=red!50, inner sep=2pt]

% The actual calendar is now rather easy:


%%%%%%%%%%%%%%%%% pgf-pie %%%%%%%%%%%%%%%%%%%%%

% https://tex.stackexchange.com/questions/433182/pgf-pie-dimension-too-large

\makeatletter
% new key
\pgfkeys{/scale numbers/.store in=\pgfpie@scale,
         /scale numbers=1}
% macro for scaling
\newcommand*{\pgfpie@scalenumber}[1]{%
    \begingroup
    \pgfkeys{/pgf/fpu}% enable fpu only locally
    \pgfmathparse{#1*\pgfpie@scale}%
    %\pgfmathroundtozerofill{\pgfmathresult}%
    \pgfmathprintnumber{\pgfmathresult}%
    \endgroup
}
% patching the internal commands
\patchcmd{\pgfpie@slice}{\beforenumber#3}{\beforenumber\pgfpie@scalenumber{#3}}{}{}
\patchcmd{\pgfpie@slice}{\beforenumber#3}{\beforenumber\pgfpie@scalenumber{#3}}{}{}
\patchcmd{\pgfpie@square}{\beforenumber#3}{\beforenumber\pgfpie@scalenumber{#3}}{}{}
\patchcmd{\pgfpie@square}{\beforenumber#3}{\beforenumber\pgfpie@scalenumber{#3}}{}{}
\patchcmd{\pgfpie@cloud}{\beforenumber#3}{\beforenumber\pgfpie@scalenumber{#3}}{}{}
\patchcmd{\pgfpie@cloud}{\beforenumber#3}{\beforenumber\pgfpie@scalenumber{#3}}{}{}
\makeatother
